\documentclass[sigconf]{acmart}

\usepackage{booktabs} % For formal tables
\usepackage[utf8x]{inputenc}
\usepackage{ucs}
\usepackage{graphicx}
\usepackage{amsmath,amsthm}
\usepackage{amssymb}
\usepackage{url}
\usepackage{stmaryrd}
\usepackage{ifpdf}
\usepackage{mathtools}
\usepackage{semantic}

\ifpdf
\usepackage{hyperref}
\fi
\usepackage{float}
\usepackage{proof}

% Copyright
%\setcopyright{none}
\setcopyright{acmcopyright}
%\setcopyright{acmlicensed}
%\setcopyright{rightsretained}
%\setcopyright{usgov}
%\setcopyright{usgovmixed}
%\setcopyright{cagov}
%\setcopyright{cagovmixed}

\AtBeginDocument{%
  \providecommand\BibTeX{{%
    \normalfont B\kern-0.5em{\scshape i\kern-0.25em b}\kern-0.8em\TeX}}}

\setcopyright{acmcopyright}
\copyrightyear{2020}
\acmYear{2020}


\acmConference[SBLP '20]{SBLP '20: Brazilian Symposium on Programming Languages}{October 19--23, 2020}{Natal, Brazil}
\acmBooktitle{SBLP '20: Brazilian Symposium on Programming Languages,
  October 19--23, 2020, Natal, Brazil}

%%%%%%%%%%%%%%%%%%%%%%%%%%%%
%% color formatting stuff %%
%%%%%%%%%%%%%%%%%%%%%%%%%%%%

\newtheorem{Lemma}{Lemma}
\newtheorem{Theorem}{Theorem}
\theoremstyle{definition}
\newtheorem{Example}{Example}

\usepackage{color}
\newcommand{\redFG}[1]{\textcolor[rgb]{0.6,0,0}{#1}}
\newcommand{\greenFG}[1]{\textcolor[rgb]{0,0.4,0}{#1}}
\newcommand{\blueFG}[1]{\textcolor[rgb]{0,0,0.8}{#1}}
\newcommand{\orangeFG}[1]{\textcolor[rgb]{0.8,0.4,0}{#1}}
\newcommand{\purpleFG}[1]{\textcolor[rgb]{0.4,0,0.4}{#1}}
\newcommand{\yellowFG}[1]{\textcolor{yellow}{#1}}
\newcommand{\brownFG}[1]{\textcolor[rgb]{0.5,0.2,0.2}{#1}}
\newcommand{\blackFG}[1]{\textcolor[rgb]{0,0,0}{#1}}
\newcommand{\whiteFG}[1]{\textcolor[rgb]{1,1,1}{#1}}
\newcommand{\yellowBG}[1]{\colorbox[rgb]{1,1,0.2}{#1}}
\newcommand{\brownBG}[1]{\colorbox[rgb]{1.0,0.7,0.4}{#1}}

\newcommand{\ColourStuff}{
  \newcommand{\red}{\redFG}
  \newcommand{\green}{\greenFG}
  \newcommand{\blue}{\blueFG}
  \newcommand{\orange}{\orangeFG}
  \newcommand{\purple}{\purpleFG}
  \newcommand{\yellow}{\yellowFG}
  \newcommand{\brown}{\brownFG}
  \newcommand{\black}{\blackFG}
  \newcommand{\white}{\whiteFG}
}

\newcommand{\MonochromeStuff}{
  \newcommand{\red}{\blackFG}
  \newcommand{\green}{\blackFG}
  \newcommand{\blue}{\blackFG}
  \newcommand{\orange}{\blackFG}
  \newcommand{\purple}{\blackFG}
  \newcommand{\yellow}{\blackFG}
  \newcommand{\brown}{\blackFG}
  \newcommand{\black}{\blackFG}
  \newcommand{\white}{\blackFG}
}

\ColourStuff



%%%%%%%%%%%%%%%%%%%%
%% lhs2TeX stuff  %%
%%%%%%%%%%%%%%%%%%%%


%% ODER: format ==         = "\mathrel{==}"
%% ODER: format /=         = "\neq "
%
%
\makeatletter
\@ifundefined{lhs2tex.lhs2tex.sty.read}%
  {\@namedef{lhs2tex.lhs2tex.sty.read}{}%
   \newcommand\SkipToFmtEnd{}%
   \newcommand\EndFmtInput{}%
   \long\def\SkipToFmtEnd#1\EndFmtInput{}%
  }\SkipToFmtEnd

\newcommand\ReadOnlyOnce[1]{\@ifundefined{#1}{\@namedef{#1}{}}\SkipToFmtEnd}
\usepackage{amstext}
\usepackage{amssymb}
\usepackage{stmaryrd}
\DeclareFontFamily{OT1}{cmtex}{}
\DeclareFontShape{OT1}{cmtex}{m}{n}
  {<5><6><7><8>cmtex8
   <9>cmtex9
   <10><10.95><12><14.4><17.28><20.74><24.88>cmtex10}{}
\DeclareFontShape{OT1}{cmtex}{m}{it}
  {<-> ssub * cmtt/m/it}{}
\newcommand{\texfamily}{\fontfamily{cmtex}\selectfont}
\DeclareFontShape{OT1}{cmtt}{bx}{n}
  {<5><6><7><8>cmtt8
   <9>cmbtt9
   <10><10.95><12><14.4><17.28><20.74><24.88>cmbtt10}{}
\DeclareFontShape{OT1}{cmtex}{bx}{n}
  {<-> ssub * cmtt/bx/n}{}
\newcommand{\tex}[1]{\text{\texfamily#1}}	% NEU

\newcommand{\Sp}{\hskip.33334em\relax}


\newcommand{\Conid}[1]{\mathit{#1}}
\newcommand{\Varid}[1]{\mathit{#1}}
\newcommand{\anonymous}{\kern0.06em \vbox{\hrule\@width.5em}}
\newcommand{\plus}{\mathbin{+\!\!\!+}}
\newcommand{\bind}{\mathbin{>\!\!\!>\mkern-6.7mu=}}
\newcommand{\rbind}{\mathbin{=\mkern-6.7mu<\!\!\!<}}% suggested by Neil Mitchell
\newcommand{\sequ}{\mathbin{>\!\!\!>}}
\renewcommand{\leq}{\leqslant}
\renewcommand{\geq}{\geqslant}
\usepackage{polytable}

%mathindent has to be defined
\@ifundefined{mathindent}%
  {\newdimen\mathindent\mathindent\leftmargini}%
  {}%

\def\resethooks{%
  \global\let\SaveRestoreHook\empty
  \global\let\ColumnHook\empty}
\newcommand*{\savecolumns}[1][default]%
  {\g@addto@macro\SaveRestoreHook{\savecolumns[#1]}}
\newcommand*{\restorecolumns}[1][default]%
  {\g@addto@macro\SaveRestoreHook{\restorecolumns[#1]}}
\newcommand*{\aligncolumn}[2]%
  {\g@addto@macro\ColumnHook{\column{#1}{#2}}}

\resethooks

\newcommand{\onelinecommentchars}{\quad-{}- }
\newcommand{\commentbeginchars}{\enskip\{-}
\newcommand{\commentendchars}{-\}\enskip}

\newcommand{\visiblecomments}{%
  \let\onelinecomment=\onelinecommentchars
  \let\commentbegin=\commentbeginchars
  \let\commentend=\commentendchars}

\newcommand{\invisiblecomments}{%
  \let\onelinecomment=\empty
  \let\commentbegin=\empty
  \let\commentend=\empty}

\visiblecomments

\newlength{\blanklineskip}
\setlength{\blanklineskip}{0.66084ex}

\newcommand{\hsindent}[1]{\quad}% default is fixed indentation
\let\hspre\empty
\let\hspost\empty
\newcommand{\NB}{\textbf{NB}}
\newcommand{\Todo}[1]{$\langle$\textbf{To do:}~#1$\rangle$}

\EndFmtInput
\makeatother
%
%
%
%
%
%
% This package provides two environments suitable to take the place
% of hscode, called "plainhscode" and "arrayhscode". 
%
% The plain environment surrounds each code block by vertical space,
% and it uses \abovedisplayskip and \belowdisplayskip to get spacing
% similar to formulas. Note that if these dimensions are changed,
% the spacing around displayed math formulas changes as well.
% All code is indented using \leftskip.
%
% Changed 19.08.2004 to reflect changes in colorcode. Should work with
% CodeGroup.sty.
%
\ReadOnlyOnce{polycode.fmt}%
\makeatletter

\newcommand{\hsnewpar}[1]%
  {{\parskip=0pt\parindent=0pt\par\vskip #1\noindent}}

% can be used, for instance, to redefine the code size, by setting the
% command to \small or something alike
\newcommand{\hscodestyle}{}

% The command \sethscode can be used to switch the code formatting
% behaviour by mapping the hscode environment in the subst directive
% to a new LaTeX environment.

\newcommand{\sethscode}[1]%
  {\expandafter\let\expandafter\hscode\csname #1\endcsname
   \expandafter\let\expandafter\endhscode\csname end#1\endcsname}

% "compatibility" mode restores the non-polycode.fmt layout.

\newenvironment{compathscode}%
  {\par\noindent
   \advance\leftskip\mathindent
   \hscodestyle
   \let\\=\@normalcr
   \let\hspre\(\let\hspost\)%
   \pboxed}%
  {\endpboxed\)%
   \par\noindent
   \ignorespacesafterend}

\newcommand{\compaths}{\sethscode{compathscode}}

% "plain" mode is the proposed default.
% It should now work with \centering.
% This required some changes. The old version
% is still available for reference as oldplainhscode.

\newenvironment{plainhscode}%
  {\hsnewpar\abovedisplayskip
   \advance\leftskip\mathindent
   \hscodestyle
   \let\hspre\(\let\hspost\)%
   \pboxed}%
  {\endpboxed%
   \hsnewpar\belowdisplayskip
   \ignorespacesafterend}

\newenvironment{oldplainhscode}%
  {\hsnewpar\abovedisplayskip
   \advance\leftskip\mathindent
   \hscodestyle
   \let\\=\@normalcr
   \(\pboxed}%
  {\endpboxed\)%
   \hsnewpar\belowdisplayskip
   \ignorespacesafterend}

% Here, we make plainhscode the default environment.

\newcommand{\plainhs}{\sethscode{plainhscode}}
\newcommand{\oldplainhs}{\sethscode{oldplainhscode}}
\plainhs

% The arrayhscode is like plain, but makes use of polytable's
% parray environment which disallows page breaks in code blocks.

\newenvironment{arrayhscode}%
  {\hsnewpar\abovedisplayskip
   \advance\leftskip\mathindent
   \hscodestyle
   \let\\=\@normalcr
   \(\parray}%
  {\endparray\)%
   \hsnewpar\belowdisplayskip
   \ignorespacesafterend}

\newcommand{\arrayhs}{\sethscode{arrayhscode}}

% The mathhscode environment also makes use of polytable's parray 
% environment. It is supposed to be used only inside math mode 
% (I used it to typeset the type rules in my thesis).

\newenvironment{mathhscode}%
  {\parray}{\endparray}

\newcommand{\mathhs}{\sethscode{mathhscode}}

% texths is similar to mathhs, but works in text mode.

\newenvironment{texthscode}%
  {\(\parray}{\endparray\)}

\newcommand{\texths}{\sethscode{texthscode}}

% The framed environment places code in a framed box.

\def\codeframewidth{\arrayrulewidth}
\RequirePackage{calc}

\newenvironment{framedhscode}%
  {\parskip=\abovedisplayskip\par\noindent
   \hscodestyle
   \arrayrulewidth=\codeframewidth
   \tabular{@{}|p{\linewidth-2\arraycolsep-2\arrayrulewidth-2pt}|@{}}%
   \hline\framedhslinecorrect\\{-1.5ex}%
   \let\endoflinesave=\\
   \let\\=\@normalcr
   \(\pboxed}%
  {\endpboxed\)%
   \framedhslinecorrect\endoflinesave{.5ex}\hline
   \endtabular
   \parskip=\belowdisplayskip\par\noindent
   \ignorespacesafterend}

\newcommand{\framedhslinecorrect}[2]%
  {#1[#2]}

\newcommand{\framedhs}{\sethscode{framedhscode}}

% The inlinehscode environment is an experimental environment
% that can be used to typeset displayed code inline.

\newenvironment{inlinehscode}%
  {\(\def\column##1##2{}%
   \let\>\undefined\let\<\undefined\let\\\undefined
   \newcommand\>[1][]{}\newcommand\<[1][]{}\newcommand\\[1][]{}%
   \def\fromto##1##2##3{##3}%
   \def\nextline{}}{\) }%

\newcommand{\inlinehs}{\sethscode{inlinehscode}}

% The joincode environment is a separate environment that
% can be used to surround and thereby connect multiple code
% blocks.

\newenvironment{joincode}%
  {\let\orighscode=\hscode
   \let\origendhscode=\endhscode
   \def\endhscode{\def\hscode{\endgroup\def\@currenvir{hscode}\\}\begingroup}
   %\let\SaveRestoreHook=\empty
   %\let\ColumnHook=\empty
   %\let\resethooks=\empty
   \orighscode\def\hscode{\endgroup\def\@currenvir{hscode}}}%
  {\origendhscode
   \global\let\hscode=\orighscode
   \global\let\endhscode=\origendhscode}%

\makeatother
\EndFmtInput
%

\DeclareMathAlphabet{\mathkw}{OT1}{cmss}{bx}{n}

\newcommand{\D}[1]{\blue{\mathsf{#1}}}
\newcommand{\Con}[1]{\red{\mathsf{#1}}}
\newcommand{\F}[1]{\green{\mathsf{#1}}}
\newcommand{\V}[1]{\purple{\mathit{#1}}}


\begin{document}

\title{A Certified Interpreter for the List Machine Benchmark}

\author{Samuel Feitosa}
\authornotemark[1]
\email{samuel.feitosa@ifsc.edu.br}
\affiliation{%
  \institution{Departamento de Informática}
  \streetaddress{Instituto Federal de Santa Catarina}
  \city{Caçador}
  \state{Santa Catarina}
  \country{Brazil}
}

\author{Rodrigo Ribeiro}
\email{rodrigo.ribeiro@ufop.edu.br}
\affiliation{%
  \institution{Prog. Pós Graduação em Ciência da Computação}
  \streetaddress{Universidade Federal de Ouro Preto}
  \city{Ouro Preto}
  \state{Minas Gerais}
  \country{Brazil}}


\begin{abstract}
Formal models are an important tool in the programming languages research
community. However, such models are full of intricacies and, due to that,
they are subject to subtle errors. Such failures motivated the usage of
tools to ensure the correctness of these formalisms. One way to eliminate
such errors is to encode models in a dependently-typed language in order
to ensure ``correctness-by-construction''.

In this work, we use follow this idea to build the verified interpreter
for the list machine benchmark in the Agda programming language.
\end{abstract}

\begin{CCSXML}\begin{hscode}\SaveRestoreHook
\column{B}{@{}>{\hspre}l<{\hspost}@{}}%
\column{E}{@{}>{\hspre}l<{\hspost}@{}}%
\>[B]{}\V{ccs2012>}{}\<[E]%
\\
\>[B]{}\V{concept>}{}\<[E]%
\\
\>[B]{}\V{concept\char95 id>10011007.10011006.10011039.10011311</concept\char95 id>}{}\<[E]%
\\
\>[B]{}\V{concept\char95 desc>Software}\;\V{and}\;\V{its}\;\V{engineering\char126 Semantics</concept\char95 desc>}{}\<[E]%
\\
\>[B]{}\V{concept\char95 significance>500</concept\char95 significance>}{}\<[E]%
\\
\>[B]{}\V{/concept>}{}\<[E]%
\\
\>[B]{}\V{concept>}{}\<[E]%
\\
\>[B]{}\V{concept\char95 id>10011007.10011006.10011041.10010943</concept\char95 id>}{}\<[E]%
\\
\>[B]{}\V{concept\char95 desc>Software}\;\V{and}\;\V{its}\;\V{engineering\char126 Interpreters</concept\char95 desc>}{}\<[E]%
\\
\>[B]{}\V{concept\char95 significance>500</concept\char95 significance>}{}\<[E]%
\\
\>[B]{}\V{/concept>}{}\<[E]%
\\
\>[B]{}\V{concept>}{}\<[E]%
\\
\>[B]{}\V{concept\char95 id>10003752.10003790.10011740</concept\char95 id>}{}\<[E]%
\\
\>[B]{}\V{concept\char95 desc>Theory}\;\V{of}\;\V{computation\char126 Type}\;\V{theory</concept\char95 desc>}{}\<[E]%
\\
\>[B]{}\V{concept\char95 significance>500</concept\char95 significance>}{}\<[E]%
\\
\>[B]{}\V{/concept>}{}\<[E]%
\\
\>[B]{}\V{/ccs2012>}{}\<[E]%
\ColumnHook
\end{hscode}\resethooks
\end{CCSXML}

\ccsdesc[500]{Software and its engineering~Semantics}
\ccsdesc[500]{Software and its engineering~Interpreters}
\ccsdesc[500]{Theory of computation~Type theory}

\keywords{Dependent types, formal semantics}

\maketitle



\section{Introduction}

The development of new programming language design, linguistic construct
or type system involves its careful formalization in order to express
its core ideas in a concise way. However, such models have many details
and complexities which hinders its correctness assurances.
Because of such problems, the programming languages research community
started to use tools, like proof assistants~\cite{Stump16,Chlipala13},
and benchmark problems to validate them and stress its suitability for
such tasks~\cite{Aydemir05,Pientka18,Appel07}.


The rest of this paper is organized as follows. Section~\ref{sec:background}
presents a brief introduction to Agda and reviews the list machine benchmark.
In Section~\ref{sec:typing}, we
describe the type system for PEGs and its relation with the original
well-formedness predicate proposed by Ford~\cite{Ford2004}.
Section~\ref{sec:interpreter} describes the intrinsically typed syntax for PEGs
and its interpreter. Limitations of our approach are discussed
in Section~\ref{sec:discussion}. Related work is discussed in
Section~\ref{sec:related}, and Section~\ref{sec:conclusion} concludes.

All the source code in this article has been formalized in Agda
version 2.6.0 using Standard Library 1.0. All source code produced,
including the \LaTeX~ source of this
article, are available on-line~\cite{peg-rep}.


\section{Background}\label{sec:background}

\paragraph{An Overview of Agda}
Agda is a dependently-typed functional programming language based on
Martin-L\"of intuitionistic type theory~\cite{Lof98}.  Function types
and an infinite hierarchy of types of types, \ensuremath{\D{Set}\;\V{l}}, where \ensuremath{\V{l}} is a
natural number, are built-in. Everything else is a user-defined
type. The type \ensuremath{\D{Set}}, also known as \ensuremath{\D{Set}_{\D{0}}}, is the type of all
``small'' types, such as \ensuremath{\D{Bool}}, \ensuremath{\D{String}} and \ensuremath{\D{List}\;\D{Bool}}.  The type
\ensuremath{\D{Set}_{\D{1}}} is the type of \ensuremath{\D{Set}} and ``others like it'', such as \ensuremath{\D{Set}\;\to \;\D{Bool}}, \ensuremath{\D{String}\;\to \;\D{Set}}, and \ensuremath{\D{Set}\;\to \;\D{Set}}. We have that \ensuremath{\D{Set}\;\V{l}} is an
element of the type \ensuremath{\D{Set}\;(\V{l+1})}, for every $l \geq 0$. This
stratification of types is used to keep Agda consistent as a logical
theory~\cite{Sorensen2006}.

An ordinary (non-dependent) function type is written \ensuremath{\V{A}\;\to \;\V{B}} and a
dependent one is written \ensuremath{(\V{x}\;\mathbin{:}\;\V{A})\;\to \;\V{B}}, where type \ensuremath{\V{B}} depends on
\ensuremath{\V{x}}, or \ensuremath{\D{\forall}\;(\V{x}\;\mathbin{:}\;\V{A})\;\to \;\V{B}}. Agda allows the definition of \emph{implicit
parameters}, i.e.,  parameters whose values can be inferred from the
context, by surrounding them in curly braces: \ensuremath{\D{\forall}\;\{\mskip1.5mu \V{x}\;\mathbin{:}\;\V{A}\mskip1.5mu\}\;\to \;\V{B}}. To
avoid clutter, we'll omit implicit arguments from the source code
presentation. The reader can safely assume that every free variable in
a type is an implicit parameter.

As an example of Agda code, consider the following data type of
length-indexed lists, also known as vectors.

\begin{hscode}\SaveRestoreHook
\column{B}{@{}>{\hspre}l<{\hspost}@{}}%
\column{3}{@{}>{\hspre}l<{\hspost}@{}}%
\column{5}{@{}>{\hspre}l<{\hspost}@{}}%
\column{9}{@{}>{\hspre}l<{\hspost}@{}}%
\column{E}{@{}>{\hspre}l<{\hspost}@{}}%
\>[3]{}\mathkw{data}\;\D{\mathbb{N}}\;\mathbin{:}\;\D{Set}\;\mathkw{where}{}\<[E]%
\\
\>[3]{}\hsindent{2}{}\<[5]%
\>[5]{}\Con{zero}\;\mathbin{:}\;\D{\mathbb{N}}{}\<[E]%
\\
\>[3]{}\hsindent{2}{}\<[5]%
\>[5]{}\Con{suc}\;\mathbin{:}\;\D{\mathbb{N}}\;\to \;\D{\mathbb{N}}{}\<[E]%
\\[\blanklineskip]%
\>[3]{}\mathkw{data}\;\D{Vec}\;(\V{A}\;\mathbin{:}\;\D{Set})\;\mathbin{:}\;\D{\mathbb{N}}\;\to \;\D{Set}\;\mathkw{where}{}\<[E]%
\\
\>[3]{}\hsindent{2}{}\<[5]%
\>[5]{}\Con{\lbrack\:\rbrack}\;{}\<[9]%
\>[9]{}\mathbin{:}\;\D{Vec}\;\V{A}\;\Con{zero}{}\<[E]%
\\
\>[3]{}\hsindent{2}{}\<[5]%
\>[5]{}\Con{\_::\_}\;\mathbin{:}\;\D{\forall}\;\{\mskip1.5mu \V{n}\mskip1.5mu\}\;\to \;\V{A}\;\to \;\D{Vec}\;\V{A}\;\V{n}\;\to \;\D{Vec}\;\V{A}\;(\Con{suc}\;\V{n}){}\<[E]%
\ColumnHook
\end{hscode}\resethooks
Constructor \ensuremath{\Con{\lbrack\:\rbrack}} builds empty vectors. The cons-operator (\ensuremath{\Con{\_::\_}})
inserts a new element in front of a vector of $n$ elements (of type
\ensuremath{\D{Vec}\;\V{A}\;\V{n}}) and returns a value of type \ensuremath{\D{Vec}\;\V{A}\;(\Con{suc}\;\V{n})}. The
\ensuremath{\D{Vec}} datatype is an example of a dependent type, i.e., a type
that uses a value (that denotes its length). The usefulness of
dependent types can be illustrated with the definition of a safe list
head function: \ensuremath{\F{head}} can be defined to accept only non-empty
vectors, i.e.,~values of type \ensuremath{\D{Vec}\;\V{A}\;(\Con{suc}\;\V{n})}.
\begin{hscode}\SaveRestoreHook
\column{B}{@{}>{\hspre}l<{\hspost}@{}}%
\column{3}{@{}>{\hspre}l<{\hspost}@{}}%
\column{E}{@{}>{\hspre}l<{\hspost}@{}}%
\>[3]{}\F{head}\;\mathbin{:}\;\D{Vec}\;\V{A}\;(\Con{suc}\;\V{n})\;\to \;\V{A}{}\<[E]%
\\
\>[3]{}\F{head}\;(\V{x}\;\Con{::}\;\V{xs})\;\mathrel{=}\;\V{x}{}\<[E]%
\ColumnHook
\end{hscode}\resethooks
In \ensuremath{\F{head}}'s definition, constructor \ensuremath{\Con{\lbrack\:\rbrack}} is not used. The
Agda type-checker can figure out, from \ensuremath{\F{head}}'s parameter type,
that argument \ensuremath{\Con{\lbrack\:\rbrack}} to \ensuremath{\F{head}} is not type-correct.



Another useful data type is the finite type, \ensuremath{\D{Fin}}\footnote{Note that Agda supports the overloading of data type constructor names.
Constructor \ensuremath{\Con{zero}} can refer to type \ensuremath{\D{\mathbb{N}}} or \ensuremath{\D{Fin}}, depending on the
context where the name is used.}, which is defined in
Agda's standard library as:

\begin{hscode}\SaveRestoreHook
\column{B}{@{}>{\hspre}l<{\hspost}@{}}%
\column{3}{@{}>{\hspre}l<{\hspost}@{}}%
\column{5}{@{}>{\hspre}l<{\hspost}@{}}%
\column{E}{@{}>{\hspre}l<{\hspost}@{}}%
\>[3]{}\mathkw{data}\;\D{Fin}\;\mathbin{:}\;\D{\mathbb{N}}\;\to \;\D{Set}\;\mathkw{where}{}\<[E]%
\\
\>[3]{}\hsindent{2}{}\<[5]%
\>[5]{}\Con{zero}\;\mathbin{:}\;\D{\forall}\;\{\mskip1.5mu \V{n}\mskip1.5mu\}\;\to \;\D{Fin}\;(\Con{suc}\;\V{n}){}\<[E]%
\\
\>[3]{}\hsindent{2}{}\<[5]%
\>[5]{}\Con{suc}\;\mathbin{:}\;\D{\forall}\;\{\mskip1.5mu \V{n}\mskip1.5mu\}\;\to \;\D{Fin}\;\V{n}\;\to \;\D{Fin}\;(\Con{suc}\;\V{n}){}\<[E]%
\ColumnHook
\end{hscode}\resethooks
Type \ensuremath{\D{Fin}\;\V{n}} has exactly \ensuremath{\V{n}} inhabitants
(elements), i.e., it is isomorphic to the set $\{0,...,n - 1\}$.
An application of such type is to define a safe vector
lookup function, which avoids the access of invalid positions.
\begin{hscode}\SaveRestoreHook
\column{B}{@{}>{\hspre}l<{\hspost}@{}}%
\column{3}{@{}>{\hspre}l<{\hspost}@{}}%
\column{E}{@{}>{\hspre}l<{\hspost}@{}}%
\>[3]{}\F{lookup}\;\mathbin{:}\;\D{\forall}\;\{\mskip1.5mu \V{A}\;\V{n}\mskip1.5mu\}\;\to \;\D{Fin}\;\V{n}\;\to \;\D{Vec}\;\V{A}\;\V{n}\;\to \;\V{A}{}\<[E]%
\\
\>[3]{}\F{lookup}\;\Con{zero}\;(\V{x}\;\Con{::}\;\anonymous )\;\mathrel{=}\;\V{x}{}\<[E]%
\\
\>[3]{}\F{lookup}\;(\Con{suc}\;\V{idx})\;(\anonymous \;\Con{::}\;\V{xs})\;\mathrel{=}\;\F{lookup}\;\V{idx}\;\V{xs}{}\<[E]%
\ColumnHook
\end{hscode}\resethooks
Thanks to the propositions-as-types principle,\footnote{It is also known as
  Curry-Howard ``isomorphism''~\cite{Sorensen2006}.} we can interpret
types as logical formulas and terms as proofs. An example is the
representation of equality as the following Agda type:

\begin{hscode}\SaveRestoreHook
\column{B}{@{}>{\hspre}l<{\hspost}@{}}%
\column{3}{@{}>{\hspre}l<{\hspost}@{}}%
\column{5}{@{}>{\hspre}l<{\hspost}@{}}%
\column{E}{@{}>{\hspre}l<{\hspost}@{}}%
\>[3]{}\mathkw{data}\;\D{\_ \equiv \_}\;\{\mskip1.5mu \V{l}\mskip1.5mu\}\;\{\mskip1.5mu \V{A}\;\mathbin{:}\;\D{Set}\;\V{l}\mskip1.5mu\}\;(\V{x}\;\mathbin{:}\;\V{A})\;\mathbin{:}\;\V{A}\;\to \;\D{Set}\;\mathkw{where}{}\<[E]%
\\
\>[3]{}\hsindent{2}{}\<[5]%
\>[5]{}\Con{refl}\;\mathbin{:}\;\V{x}\;\D{\equiv}\;\V{x}{}\<[E]%
\ColumnHook
\end{hscode}\resethooks

This type is called propositional equality. It defines that there is
a unique evidence for equality, constructor \ensuremath{\Con{refl}} (for reflexivity),
that asserts that the only value equal to \ensuremath{\V{x}} is itself. Given a predicate \ensuremath{\V{P}\;\mathbin{:}\;\V{A}\;\to \;\D{Set}}
and a vector \ensuremath{\V{xs}}, the type \ensuremath{\D{All}\;\V{P}\;\V{xs}} is used to build proofs that \ensuremath{\V{P}} holds for all
elements in \ensuremath{\V{xs}} and it is defined as:
\begin{hscode}\SaveRestoreHook
\column{B}{@{}>{\hspre}l<{\hspost}@{}}%
\column{3}{@{}>{\hspre}l<{\hspost}@{}}%
\column{6}{@{}>{\hspre}l<{\hspost}@{}}%
\column{41}{@{}>{\hspre}l<{\hspost}@{}}%
\column{E}{@{}>{\hspre}l<{\hspost}@{}}%
\>[3]{}\mathkw{data}\;\D{All}\;(\V{P}\;\mathbin{:}\;\V{A}\;\to \;\D{Set})\;\mathbin{:}\;\D{Vec}\;\V{A}\;\V{n}\;\to \;{}\<[41]%
\>[41]{}\D{Set}\;\mathkw{where}{}\<[E]%
\\
\>[3]{}\hsindent{3}{}\<[6]%
\>[6]{}\Con{\lbrack\:\rbrack}\;\mathbin{:}\;\D{All}\;\V{P}\;\Con{\lbrack\:\rbrack}{}\<[E]%
\\
\>[3]{}\hsindent{3}{}\<[6]%
\>[6]{}\Con{\_::\_}\;\mathbin{:}\;\D{\forall}\;\{\mskip1.5mu \V{x}\;\V{xs}\mskip1.5mu\}\;\to \;\V{P}\;\V{x}\;\to \;\D{All}\;\V{P}\;\V{xs}\;\to \;\D{All}\;\V{P}\;(\V{x}\;\Con{::}\;\V{xs}){}\<[E]%
\ColumnHook
\end{hscode}\resethooks
The first constructor specifies that \ensuremath{\D{All}\;\V{P}} holds for the empty vector and
constructor \ensuremath{\Con{\_::\_}} builds a proof of \ensuremath{\D{All}\;\V{P}\;(\V{x}\;\Con{::}\;\V{xs})} from proofs of
\ensuremath{\V{P}\;\V{x}} and \ensuremath{\D{All}\;\V{P}\;\V{xs}}. Since this type has the same structure of vectors,
some functions on \ensuremath{\D{Vec}} have similar definitions for type \ensuremath{\D{All}}. As an example
used in our formalization, consider the function \ensuremath{\F{lookup}}, which extracts a
proof of \ensuremath{\V{P}} for the element at position \ensuremath{\V{v}\;\mathbin{:}\;\D{Fin}\;\V{n}} in a \ensuremath{\D{Vec}}:
\begin{hscode}\SaveRestoreHook
\column{B}{@{}>{\hspre}l<{\hspost}@{}}%
\column{4}{@{}>{\hspre}l<{\hspost}@{}}%
\column{E}{@{}>{\hspre}l<{\hspost}@{}}%
\>[4]{}\F{lookup}\;\mathbin{:}\;\{\mskip1.5mu \V{xs}\;\mathbin{:}\;\D{Vec}\;\V{A}\;\V{n}\mskip1.5mu\}\;\to \;\D{Fin}\;\V{n}\;\to \;\D{All}\;\V{P}\;\V{xs}\;\to \;\V{P}\;\V{x}{}\<[E]%
\\
\>[4]{}\F{lookup}\;\Con{zero}\;(\V{px}\;\Con{::}\;\anonymous )\;\mathrel{=}\;\V{px}{}\<[E]%
\\
\>[4]{}\F{lookup}\;(\Con{suc}\;\V{idx})\;(\anonymous \;\Con{::}\;\V{pxs})\;\mathrel{=}\;\F{lookup}\;\V{idx}\;\V{pxs}{}\<[E]%
\ColumnHook
\end{hscode}\resethooks
An important application of dependent types is to encode programming languages
syntax. The role of dependent types in this domain is to encode programs that
only allow well-typed and well-scoped terms~\cite{Benton2012}. Intuitively, we encode
the static semantics of the object language in the host language AST's
constructor, leaving the responsibility of checking type safety to the
host's language type checker. As an example, consider the following simple
expression language.
\begin{hscode}\SaveRestoreHook
\column{B}{@{}>{\hspre}l<{\hspost}@{}}%
\column{4}{@{}>{\hspre}l<{\hspost}@{}}%
\column{7}{@{}>{\hspre}l<{\hspost}@{}}%
\column{E}{@{}>{\hspre}l<{\hspost}@{}}%
\>[4]{}\mathkw{data}\;\D{Expr}\;\mathbin{:}\;\D{Set}\;\mathkw{where}{}\<[E]%
\\
\>[4]{}\hsindent{3}{}\<[7]%
\>[7]{}\Con{True}\;\Con{False}\;\mathbin{:}\;\D{Expr}{}\<[E]%
\\
\>[4]{}\hsindent{3}{}\<[7]%
\>[7]{}\Con{Num}\;\mathbin{:}\;\D{\mathbb{N}}\;\to \;\D{Expr}{}\<[E]%
\\
\>[4]{}\hsindent{3}{}\<[7]%
\>[7]{}\Con{\_\land\_}\;\Con{\_+\_}\;\mathbin{:}\;\D{Expr}\;\to \;\D{Expr}\;\to \;\D{Expr}{}\<[E]%
\ColumnHook
\end{hscode}\resethooks
Using this data type,\footnote{Agda supports the definition of mixfix operators.
We can use underscores to mark arguments positions.} we can construct expressions
that denote terms that should not be considered well-typed like
\ensuremath{(\Con{Num}\;\V{1})\;\F{+}\;\Con{True}}. Using this approach, we need to specify the static semantics
as another definition, which should consider all possible cases to avoid the
definition of ill-typed terms.

A better approach is to combine the static semantics and language syntax into
a single definition, as shown below.

\begin{hscode}\SaveRestoreHook
\column{B}{@{}>{\hspre}l<{\hspost}@{}}%
\column{4}{@{}>{\hspre}l<{\hspost}@{}}%
\column{7}{@{}>{\hspre}l<{\hspost}@{}}%
\column{E}{@{}>{\hspre}l<{\hspost}@{}}%
\>[4]{}\mathkw{data}\;\D{Ty}\;\mathbin{:}\;\D{Set}\;\mathkw{where}{}\<[E]%
\\
\>[4]{}\hsindent{3}{}\<[7]%
\>[7]{}\Con{Nat}\;\Con{Bool}\;\mathbin{:}\;\D{Ty}{}\<[E]%
\\[\blanklineskip]%
\>[4]{}\mathkw{data}\;\D{Expr}\;\mathbin{:}\;\D{Ty}\;\to \;\D{Set}\;\mathkw{where}{}\<[E]%
\\
\>[4]{}\hsindent{3}{}\<[7]%
\>[7]{}\Con{True}\;\Con{False}\;\mathbin{:}\;\D{Expr}\;\Con{Bool}{}\<[E]%
\\
\>[4]{}\hsindent{3}{}\<[7]%
\>[7]{}\Con{Num}\;\mathbin{:}\;\Con{Nat}\;\to \;\D{Expr}\;\Con{Nat}{}\<[E]%
\\
\>[4]{}\hsindent{3}{}\<[7]%
\>[7]{}\Con{\_\land\_}\;\mathbin{:}\;\D{Expr}\;\Con{Bool}\;\to \;\D{Expr}\;\Con{Bool}\;\to \;\D{Expr}\;\Con{Bool}{}\<[E]%
\\
\>[4]{}\hsindent{3}{}\<[7]%
\>[7]{}\Con{\_+\_}\;\mathbin{:}\;\D{Expr}\;\Con{Nat}\;\to \;\D{Expr}\;\Con{Nat}\;\to \;\D{Expr}\;\Con{Nat}{}\<[E]%
\ColumnHook
\end{hscode}\resethooks

In this definition, the \ensuremath{\D{Expr}} type is indexed by a value of type \ensuremath{\D{Ty}} which
indicates the type of the expression being built. In this approach, Agda's
type system can enforce that only well-typed terms could be written.
%A definition which uses the expression |(Num 1) + True| will be rejected by Agda's type checker automatically.
Agda's type checker will automatically reject a definition which uses the expression \ensuremath{(\Con{Num}\;\V{1})\;\F{+}\;\Con{True}}.

%Dependently typed pattern matching is built by using the so-called
%|with| construct, that allows for matching intermediate
%values~\cite{McBride2004}. If the matched value has a dependent type,
%then its result can affect the form of other values. For example,
%consider the following code that defines a type for natural number
%parity. If the natural number is even, it can be represented as the
%sum of two equal natural numbers; if it is odd, it is equal to one
%plus the sum of two equal values. Pattern matching on a value of
%|Parity n| allows to discover if $n = j + j$ or $n = S (k + k)$,
%for some $j$ and $k$ in each branch of |with|.  Note that the
%value of $n$ is specialized accordingly, using information ``learned''
%by the type-checker.
%\begin{spec}
%data Parity : Nat -> Set where
%   Even : forall {n : Nat} -> Parity (n + n)
%   Odd  : forall {n : Nat} -> Parity (S (n + n))
%
%parity : (n : Nat) -> Parity n
%parity = -- definition omitted

%natToBin : Nat -> List Bool
%natToBin zero = []
%natToBin k with (parity k)
%   natToBin (j + j)     | Even = false :: natToBin j
%   natToBin (succ (j + j)) | Odd  = true  :: natToBin j
%\end{spec}

For further information about Agda, see~\cite{Norell2009,Stump16}.

\paragraph{The List Machine Benchmark}

\begin{equation}
\inference{}
          {(r, (\iota_1;\iota_2);\iota_3) \xmapsto{p} (r, \iota_1;(\iota_2;\iota_3))}[step-seq]
\end{equation}

\begin{equation}
\inference{r(v) = cons(a_0,a_1)~~~r[v':=a_0]=r'}
          {(r, (\textbf{fetch-field}~v~0~v';\iota)) \xmapsto{p} (r', \iota)}[step-fetch-field-0]
\end{equation}

\begin{equation}
\inference{r(v) = cons(a_0,a_1)~~~r[v':=a_1]=r'}
          {(r, (\textbf{fetch-field}~v~1~v';\iota)) \xmapsto{p} (r', \iota)}[step-fetch-field-1]
\end{equation}

\begin{equation}
\inference{r(v_0) = a_0~~~r(v_1)=a_1~~~r[v':=cons(a_0,a_1)]=r'}
          {(r, (\textbf{cons}~v_0~v_1~v';\iota)) \xmapsto{p} (r', \iota)}[step-cons]
\end{equation}

\begin{equation}
\inference{r(v) = cons(a_0,a_1)}
          {(r, (\textbf{branch-if-nil}~v~l;\iota)) \xmapsto{p} (r, \iota)}[step-branch-not-taken]
\end{equation}

\begin{equation}
\inference{r(v) = nil~~~p(l)=\iota'}
          {(r, (\textbf{branch-if-nil}~v~l;\iota)) \xmapsto{p} (r, \iota')}[step-branch-taken]
\end{equation}

\begin{equation}
\inference{p(l)=\iota'}
          {(r, \textbf{jump}~l) \xmapsto{p} (r, \iota')}[step-jump]
\end{equation}

\begin{equation}
\inference{(r, \iota) \xmapsto{p} (r', \iota')~~~(p, r', \iota')}
          {(p, r, \iota) \Downarrow}[run-step]
\end{equation}

\begin{equation}
\inference{}
          {(p, r, \textbf{halt}) \Downarrow}[run-halt]
\end{equation}

\begin{equation}
\inference{\{\}[\textbf{v}_0:=\text{nil}]=r~~~p(\textbf{L}_0)=\iota~~~(p,r,\iota)\Downarrow}
          {p \Downarrow}[run-prog]
\end{equation}

%\subsection{Typing}

%\subsubsection{Subtyping}

\begin{equation}
\inference{}
          {\tau \subset \tau}[subtype-refl]
\end{equation}

\begin{equation}
\inference{}
          {\text{nil} \subset \text{list}~\tau}[subtype-nil]
\end{equation}

\begin{equation}
\inference{\tau \subset \tau'}
          {\text{list}~\tau \subset \text{list}~\tau'}[subtype-list]
\end{equation}

\begin{equation}
\inference{\tau \subset \tau'}
          {\text{listcons}~\tau \subset \text{list}~\tau'}[subtype-listcons]
\end{equation}

\begin{equation}
\inference{\tau \subset \tau'}
          {\text{listcons}~\tau \subset \text{listcons}~\tau'}[subtype-listmixed]
\end{equation}

%\subsubsection{Instruction typing}

\begin{equation}
\inference{\Pi \vdash_{\text{instr}} \Gamma\{ \iota_1 \}\Gamma'~~~\Pi \vdash_{\text{instr}} \Gamma'\{ \iota_2 \}\Gamma''}
          {\Pi \vdash_{\text{instr}} \Gamma\{ \iota_1 ; \iota_2 \}\Gamma''}[check-instr-seq]
\end{equation}

\begin{equation}
\inference{\Gamma(v) = \text{list}~\tau~~~\Pi(l) = \Gamma_1~~~\Gamma[v:=\text{nil}]=\Gamma'~~~\Gamma' \subset \Gamma_1}
          {\Pi \vdash_{\text{instr}} \Gamma\{ \textbf{branch-if-nil}~v~l \}(v: \text{listcons}~\tau,~\Gamma')}[check-instr-branch-list]
\end{equation}

\begin{equation}
\inference{\Gamma(v) = \text{listcons}~\tau~~~\Pi(l) = \Gamma_1~~~\Gamma[v:=\text{nil}]=\Gamma'~~~\Gamma' \subset \Gamma_1}
          {\Pi \vdash_{\text{instr}} \Gamma\{ \textbf{branch-if-nil}~v~l \}\Gamma}[check-instr-branch-listcons]
\end{equation}

\begin{equation}
\inference{\Gamma(v) = \text{nil}~~~\Pi(l) = \Gamma_1~~~\Gamma \subset \Gamma_1}
          {\Pi \vdash_{\text{instr}} \Gamma\{ \textbf{branch-if-nil}~v~l \}\Gamma}[check-instr-branch-nil]
\end{equation}

\begin{equation}
\inference{\Gamma(v) = \text{listcons}~\tau~~~\Gamma[v':=\tau]=\Gamma'}
          {\Pi \vdash_{\text{instr}} \Gamma\{ \textbf{fetch-field}~v~0~v' \}\Gamma'}[check-instr-fetch-0]
\end{equation}

\begin{equation}
\inference{\Gamma(v) = \text{listcons}~\tau~~~\Gamma[v':=\text{list}~\tau]=\Gamma'}
          {\Pi \vdash_{\text{instr}} \Gamma\{ \textbf{fetch-field}~v~1~v' \}\Gamma'}[check-instr-fetch-1]
\end{equation}

\begin{equation}
\inference{\Gamma(v_0) = \tau_0~~~~~~~~~\Gamma(v_1) = \tau_1 \\ (\text{list}~\tau_0) \sqcap \tau_1=\text{list}~\tau~~~\Gamma[v:=\text{listcons}~\tau]=\Gamma'}
          {\Pi \vdash_{\text{instr}} \Gamma\{ \textbf{cons}~v_0~v_1~v \}\Gamma'}[check-instr-cons]
\end{equation}

%\subsubsection{Block typings}

\begin{equation}
\inference{}
          {\Pi;\Gamma\vdash_{\text{block}} \textbf{halt}}[check-block-halt]
\end{equation}

\begin{equation}
\inference{\Pi\vdash_{\text{instr}} \Gamma\{\iota_1\}\Gamma'~~~\Pi;\Gamma'\vdash_{\text{block}} \iota_2}
          {\Pi;\Gamma\vdash_{\text{block}} \iota_1;\iota_2}[check-block-seq]
\end{equation}

\begin{equation}
\inference{\Pi(l)=\Gamma_1~~~\Gamma \subset \Gamma_1}
          {\Pi;\Gamma\vdash_{\text{block}} \textbf{jump}~l}[check-block-jump]
\end{equation}

%\subsubsection{Program typings}

\begin{equation}
\inference{\Pi(l)=\Gamma~~~\Pi;\Gamma\vdash_{\text{block}} \iota~~~\Pi\vdash_{\text{blocks}} p}
          {\Pi\vdash_{\text{blocks}} l: \iota;~p}[check-blocks-label]
\end{equation}

\begin{equation}
\inference{}
          {\Pi\vdash_{\text{blocks}} \textbf{end}}[check-blocks-empty]
\end{equation}


\bibliographystyle{ACM-Reference-Format}
\bibliography{main}
\end{document}
\endinput
